 % !Mode:: "TeX:UTF-8"
\chapter{可见光多波段OFDM系统速率自适应技术研究}
\section{引言}
自适应传输技术在上世纪60年代就已经被提出,其基本思想就是根据实时的信道质量决定调制参数,目标是优化通信质量,但是因为其计算复杂度很大,实现困难而没有引起研究人员足够的重视\cite{徐凌峰2007ofdm},直到上世纪80年代末,人们对高速可靠的通信系统的需求越来越强烈,同时由于数字集成电路的快速发展,其计算能力已能够支撑复杂的算法,所以自适应传输计算重新进入研究人员的视野,并且成功用于DSL、WCDMA等通信系统。前面我们已经介绍过OFDM系统及其信道估计的方法,已经了解OFDM技术是把实际通信信道划分成若干个子信道,每个子信道可以认为是独立传输的,如果所有的子载波上都使用同样的调制方式,那么整个系统的误比特率性能就由那些处于深衰落处的子载波决定,如在前两章已经介绍室内可见光信道就是低通的,则此时高频的子载波决定了整个系统的性能,这样的方法显然是不合理的,所以在第三章的仿真中就使用了表所示的调制策略,但是之前得到这样的策略是由主观判断得到的,而没有充足的理论依据。本章将详细介绍OFDM自适应技术及其在可见光通信中的应用,首先将从信息论的角度探讨自适应传输的原理,然后将介绍现有的几种经典的自适应算法并且仿真分析它们的性能,最后将介绍一种可见光自适应传输方案。
\section{自适应传输的信息论基础}
通信技术经过将近一个世纪的发展,不断出现的使得系统传输速率越来越高,因此人们要问在特定的通信信道下传输速率的极限是什么?这正是经典的香农信息论已经回答的问题,同时也给系统设计者指明了要达到这个极限系统要满足的条件,虽然这些严苛的条件在实际设计中不可能完全满足,但是已有很多系统的系统已经很接近香农限了。

自适应传输是一种提高频谱利用率的通信技术,同样也满足香农信息论关于信道容量的结论,所以自适应传输技术也不可能使得实际系统传输速率突破香农限,而是在香农信息论的指导下去设计系统,使得系统逼近这一极限,因此在讨论自适应传输的具体算法之前了解一些香农信息论的知识非常必要。
\subsection{高斯信道容量}
信道容量是一个通信信道环境一个重要的度量指标,它的含义是在该信道下传输速率的上限,如果本身信道容量就很小的环境下,无论使用何种通信技术也不能实现高速通信系统,反之在信道容量大的信道环境下,可以通过精心设计系统以达到高速通信的目的。在香农信息论中,信道容量是用互信息量来描述的,其数学表达式为:

式中表示发送信号X与接收信号Y之间的互信息,p(x)、p(y)和p(x,y)分别表示X=x的概率,Y=y的概率及X=x并且Y=y的联合概率。根据上面的定义,我们可以得到在高斯信道下的信道容量公式为:
上式中B为高斯信道带宽,S为输入信号的平均功率,为单边带高斯噪声功率谱密度,表示接收信噪比。该容量是当发送信号X服从高斯分布时取得。而在实际的数字传输系统中,这个条件是无法满足的,所以该信道容量只能作为系统传输速率的上限。
\subsection{注水定理}
在无线通信环境下,由于放大器等硬件是非理想的,并且信号在传输过程中会发生散射、反射等造成多径,实际的通信系统信道远比加性高斯信道复杂得多。我们假设信道的传输函数为,输入信号的功率谱密度为,单边带高斯噪声功率密度还是。为了推到这样的信道的信道容量,可见将整个信道带宽均分为N个子信道,则每个子信道的带宽为,当N足够大时,中心频率为处的子信道可以看作是信道增益为的带限信道。于是整个信道容量等于所以子信道容量之和:
从上式中可以看出信道传输函数H(f)和发送信号的功率谱密度分布及噪声功率共同决定了信道容量的大小。假设系统发射功率受限,即:
则由拉格朗日乘子法计算可得,当S(f)的分布满足下式时可以达到信道容量。
其中K为常数,并且满足式中的功率受限条件。

式得到S(f)分布的过程也称为“注水定理”或者“注水算法”,它表示要想在传输函数为H(f),噪声功率密度为的信道下要达到信道容量S(f)的分布一定要满足上式。其物理含义是:当信噪比较大时,应该给该处的子信道分配更多的功率;反之当信噪比很小时,分配的功率也很小;甚至当信噪比过小(小于1/K)时不分配功率。
OFDM技术的基本原理就是将整个信道划分为相互正交的并行子信道,这个与前面注水算法的推导过程很相识,只是OFDM系统中子载波的数目是有限的,调制阶数是离散的。从此可以看出注水算法很容易在OFDM系统中现实,我们将在下一节中详细介绍。
\section{OFDM系统自适应算法研究}
在一个有N个子载波的OFDM系统中,假设各个子载波上的信道增益和高斯噪声方差已知,若 和分别表示分配到第k 个子载波上比特数和功率,如设置误比特率为固定值,则香农公式与有如下关系:
也可以写为:
其中表示信噪比差(SNR gap),它由误比特率BER及调制星座图决定,对于QAM调制,如不加信道编码,则与之间关系如下\cite{余官定2005ofdm}:

OFDM自适应传输在不考虑信道编码的情况下其实就是比特和功率在各个子载波上分配问题,它可以有传输速率、发射功率和误比特率三个优化目标量,这样也就有三种优化准则,即固定误比特率和功率的速率最大化(Rate Adaptive,RA)准则、固定误比特率和速率的功率最小化(Margin Adaptive, MA)准则和固定功率和速率的误比特率最小化(BER Adaptive, BA)准则。同时需要提出的是,与前面的理论分析不同,实际系统可以选择的调制方式是离散的(表示整数集),也就是说OFDM自适应其实是一个整数优化问题(Integer Programming,IP),之前在推导注水算法时用到的拉格朗日乘子法不能再用到实际系统中,需要根据实际系统的需要,根据优化准则去选择合适的优化方法,下面将介绍几种经典的OFDM自适应算法。
\subsection{Hughes-Hartogs算法}
Hughes-Hartogs算法\cite{hughes1989ensemble}受数学中“贪婪(Greedy)优化”的启发,本质上也是一种贪婪算法,由Hughes-Hartogs于1988年提出,其描述如下:假设OFDM系统中各个子载波上的信道增益和噪声都是已知的,这样在特定的BER要求下要传输1比特所需要的功率也是可知的,Hughes-Hartogs算法就是每次遍历一次所有的子载波,选择需要功率最少的子载波放置1比特数据,如此循环迭代,知道用完所有功率(RA准则)或者达到目标速率(MA准则)。对于RA 准则和MA准则,Hughes-Hartogs算法是最优的。它的具体实现步骤如下:
\begin{description}
\item{\bf{步骤1:}}令表示已用的功率,表示已分配比特。
\item{\bf{步骤2:}}遍历计算每个子载波上增加1比特所需要增加的功率:
\item{\bf{步骤3:}}找到所需增加额外功率的子载波:
对于RA准则,若则结束算法,反之则置,返回\textbf{步骤2}继续执行;对于MA准则,若或者则结束算法,反之则置返回\textbf{步骤2} 继续执行。其中表示限制功率和目标速率。
\end{description}
该算法的得到的最后结果为即为每个子载波上应该分配的比特数,如果则说明该子载波处信道质量太差,应设置为虚拟子载波不传输数据。从算法的步骤中可以看出每次迭代找到的都是最优的子载波,所以整个算法也是最优的,但是其复杂度为,对于RA准则而言,随着信噪比的增加,可传输速率也会增大,Hughes-Hartogs算法复杂度也会增加,同时子载波数的增加也会增加算法的复杂度,这也限制了Hughes-Hartogs算法在工程中的应用。
\subsection{P.S.Chow算法}
P.S.Chow算法\cite{chow1995practical}是一种RA准则下次优算法,即限定了发射功率和速率优化误比特率性能。它由P.S.Chow于1995年提出,该算法总体上分三步进行,第一步迭代找到(近似)最优系统性能门限 (该门限表示系统在能够达到目标误比特率基础上还能容忍的额外噪声,以dB为单位);第二步确定各个子载波上的比特分配,如果第一步迭代在迭代次数达到上限后还没有收敛,即,则会在这步调整使得;第三步是功率分配,首先根据根据各个子载波上的比特数和目标误比特率得到各个子载波上所需功率,然后总功率也限制功率之间的关系,得到一个功率系数,调整各个子载波上的功率,使得总功率满足功率限制。具体算法步骤如下:
\begin{description}
\item{\bf{步骤1:}}计算各个子载波上的信噪比,并且假设各个子载波上是归一化等功率的,即。
\item{\bf{步骤2:}}令和,其中表示迭代计数器,表示使用的子载波。
\item{\bf{步骤3:}}根据下面的式子,遍历所有子载波计算:
\begin{eqnarray}
b(k)=\log2(1+\frac{SNR(k)}{\Gamma+\gamma_{margin}(dB)})\\
\hat{b}(k)=round[b(k)]\\
diff(k)=b(k)-\hat{b}(k)
\end{eqnarray}
表示取整,如果,则。
\item{\bf{步骤4:}}令,若说明整个信道条件太差,无法传输数据,算法退出。
\item{\bf{步骤5:}}用下式更新:
其中表示目标速率。
\item{\bf{步骤6:}}令迭代计数器加1,。
\item{\bf{步骤7:}}若 并且 ,则令跳到\textbf{步骤3} 执行,否则跳到\textbf{步骤8}。其中表示设置的最大迭代次数。
\item{\bf{步骤8:}}若,则选择:
令,重复执行\textbf{步骤8}直到
\item{\bf{步骤9:}}若,则选择:
令,重复执行\textbf{步骤9}直到
\item{\bf{步骤10:}}根据前面得到各子载波上分配的比特,计算各个子载波上应该分配的功率使得各个子载波上的误比特率成立。这样就会改变了步骤1中的等功率分配了。
\item{\bf{步骤11:}}在步骤10中得到的各个子载波上的功率基础上再乘以一个比例因子,使得发射总功率表示额定功率。
\end{description}

P.S.Chow 算法的思想是先通过目标速率迭代寻找最优系统性能门限,迭代终止条件为分配的总速率等于目标速率,但是在某些信道条件下,总速率会在目标速率附近震荡而永远不会收敛到目标速率,因此Chow算法还设计了另一个收敛条件,即迭代次数等于初始化设置的最大迭代次数。如果是由迭代次数限制而终止迭代的,则通过步骤8、9来保证总速率等于目标速率,得到各个子载波上的比特分配,然后根据各子载波上的比特分配确定各个子载波上需要的功率,最后在各个子载波的功率前面乘以一个相同的系数,使得总功率满足功率限制条件,其最坏情况下算法复杂度为,一般情况下小于10 次就会收敛,所以相对于Hughes-Hartogs算法,其算法复杂度下降了很多。
\subsection{Fischer算法}
Fischer算法\cite{fischer1996new}也是一种固定发射功率和速率优化系统误比特率的算法(RA准则),但是它与Hughes-Hartogs算法和P.S.Chow算法不同,这两者都是从信道容量的角度出发的,而Fischer算法则是从各个子载波上的误比特率出发,算法的核心思想认为当所有被利用的子载波上的误符号率相等时,则会使得系统的误比特率最优,这个其实也比较容易理解,因为如果有某些子载波的误符号率很高的话,则这些子载波就决定了整个系统的误比特率。

Fischer是根据QAM调制的误符号率来推导的,第各子载波上的误符号率可以写为:
其中是互补高斯积分函数,分别表示第个子载波上调制星座图中的最小欧氏距离和噪声方差。要使得使得所有子载波上的误符号率都相等,也就是使得归一化信噪比等于一个常数,即:
整个优化问题就是在功率和速率的限制下最大化。又因QAM调制的符号可以表示为表示增益系数,则有,并且第个子载波上的平均功率为:
表示第个子载波上放置的比特数,利用功率限制条件有:
所以有:
要最大化,在功率和速率限制条件下,利用拉格朗日优化得到为常数,利用这个结有:
因为都相等,所以得到:
通过上式得到各个子载波上的比特分配之后,要在可用子载波集合 中去掉的子载波,并且更新可用的子载波数为,利用上式迭代,直到。因为为常数,由式可知所有可用的子载波上的能量也应该相等:

上面只是理论上的推导,没有限制为整数,但是在实际应用中必须要加上这一条件,相应的算法也有小许改动,下面给出Fischer算法的具体实现步骤:
\begin{description}
\item{\bf{步骤1:}}首先计算各子载波上的等效噪声方差(等于实际噪声方差除以信道增益)为子载波数。然后计算各子载波上的对数噪声,将这些值保存下来,之后的计算中可以重复使用;初始化。
\item{\bf{步骤2:}}计算可用的子载波上应该分配的比特数:
\item{\bf{步骤3:}}在集合中去掉所有的,并且更新,跳转到\textbf{步骤2}执行,直到。
\item{\bf{步骤4:}}量化分配的比特,,记录量化误差。
\item{\bf{步骤5:}}计算总比特数
\item{\bf{步骤6:}}若,则跳转到\textbf{步骤7},否则:\\
若,则选择量化误差最小的子载波,假设其序号为,调整,继续步骤6直到;\\
若,则选择量化误差最大的子载波,假设其序号为,调整,继续步骤6直到;
\item{\bf{步骤7:}}最后根据各个子载波上分配的比特数,按下式计算各子载波上应该分配的功率:
\end{description}

Fischer算法给出了比特和功率分配的闭式解(经有限次迭代之后一定收敛),而且算法复杂度比较低,通过在步骤1中把存储下来,在接下来的运算中只需要进行加减法和除法运算,尤其是当目标速率设置适当(步骤5中计算得到的接近)时,算法复杂度可以进一步降低,为量级,相对于P.S.Chow算法又有了降低。

上面介绍了三种非常经典的OFDM系统比特功率分配算法,也有研究人员对这些算法进行改良,如改进的贪婪算法不想原算法逐比特分配,而是先通过对分法搜索得到注水定理中的注水线(注水线的上下界是确定的\cite{余官定2005ofdm}),然后根据注水线得到各子载波上的比特数并向下取整,计算这些比特所需要的功率,最后用贪婪算法剩下的功率。这种改进的算法在高信噪比下可以大大降低复杂度。也有学者通过迭代的方式将注水算法应用到离散比特分配中,其基本思路是先令注水线等于其上界,然后按这个注水线分配比特并且四舍五入取整并计算所需的功率,如果功率满足限制条件则分配完毕,否则降低注水线重新注水,直到满足功率限制。

\subsection{仿真结果分析}

前面介绍了三种不同的比特功率分配算法,下面通过仿真来展示其性能。仿真的信道还是使用节中的可见光LOS信道和DOW信道,使用LOS信道更贴近本课题的硬件设计,而使用DOW信道模型可以增加信道的多样性,更能客观地比较不同算法之间的性能差别。仿真中使用QAM调制,并且在仿真中对最高调制阶数进行了限制(1024QAM),这个在实际工程中也是常见的,毕竟很少有系统用到1024QAM以上的调制。

图给出了在LOS信道不同比特功率分配算法的误比特率随信噪比变化的性能,为了便于比较,将目标速率设置为736 bit/OFDM signal,与表中的相等。图中可以看出,虽然在表中我们已经专门针对可见光通信的低通特意设计了比特分配,但是使用自适应算法得到比特功率分配在性能上还是有些提高的,尤其是在高信噪比的情况下。但是在固定的LOS信道下,Hughes-Hartogs算法、Chow算法和Fischer算法之间的BER性能差异很小,这也说明了Chow算法和Fischer算法在计算复杂度降低的情况下,性能损失不大。需要说明的是,Hughes-Hartogs算法在RA和MA准则下是最优的,但是这里BER 性能的比较是BA准则,Hughes-Hartogs算法在BA准则下的计算过程是先使用MA准则得到各子载波的分配的比特,然后根据计算每个子载波达到目标BER所需要的功率,最后归一化使得总功率满足功率限制,所以在BA准则下Hughes-Hartogs算法也不是最优的。图展示了各种比特功率分配算法在DOW信道模型下的性能,这里DOW信道模型采用的是节中提到的指数衰减模型和天花板反射模型混合得到信道模型,结果也LOS信道下类似,三种算法BER性能都优于表3.2 中的分配方案,但是三种算法之间的性能非常接近。

图和给出了三种算法在各个子载波上分配的比特和功率,可以看出分配的比特数随着频率(子载波系数)的增加而减少,这是因为我们的信道是低通的,给低频段高信噪比的子载波分配更多比特而在高频段低信噪比的子载波分配更少比特是合理的。各子载波上分配的功率上下波动,但是有个规律- 在分配了相同比特的子载波上,功率是随着频率递增的,这还是由于低通信道,在相同阶数的调制的子载波上为了使得BER尽可能相等,高频的子载波要分配更多的比特。综上所述仿真结果是与之前的理论分析相吻合的。

\section{可见光通信中的自适应方案研究}
前面提到的算法都是具有普适性的,即可以用在任何信道的自适应OFDM系统中,但是我们看到即使Chow算法和Fischer算法相对于Hughes-Hartogs算法进行了简化,但对于实际系统而言复杂度还是偏高的。我们可以充分利用可见光信道的特征,进一步简化比特和功率分配算法。
\subsection{SBLA算法}
从前面的分析及仿真结果可以看到,可见光信道相邻子载波之间存在明显的相关性,特别是LOS信道,信噪比几乎是随频率单调递减的,而不会出现非常明显的起伏,使得相邻子载波之间可支持的调制阶数差别很大。这样的信道条件非常适合用简单分组比特功率分配算法(Simple Blockwise Loading Algorithm,SBLA)\cite{grunheid2000adaptive},该算法的核心思想就是将所有可用的子载波划分为若干个子载波组,每个子载波组使用相同的调制阶数,不同组之间的调制阶数可以不同。这样可以带来两个方面的好处,一是降低了算法复杂度,二是减少了发射端与接收端之间交换的用于协商自适应参数的信令信息。

SBLA算法首先根据目标误比特率和AWGN下QAM各阶调制的BER与SNR之间的关系得到各阶调制的SNR门限,如图所示,我们把目标BER设置为,则可以得到如表所示的门限信噪比,有了这组门限值之后可以让每个子载波组的平时信噪比与这组门限比较,找到满足满足门限的最大调制阶数,这样就可以得到每个子载波组的初始比特分配即总速率,如果总速率等于目标速率速率则比特分配结束,否则使用类似与Fischer算法步骤6的方法进行调整,下面给出SBLA算法实现的具体步骤:
\begin{description}
\item{\bf{步骤1:}}根据总子载波数量确定分组数及每组中包含的子载波数,计算各组的平均信噪比:
其中表示第i个子载波组中的第k个子载波。
\item{\bf{步骤2:}}根据各子载波组的平均信噪比及信噪比门限,假设第i各子载波组中分配的比特数为,使用条件找到各子载波组的初始比特分配,然后计算并存储平均信噪比与该调制阶数对应的门限信噪比余量,其定义如下:
\item{\bf{步骤3:}}计算初始分配的总速率:
如果,可跳到步骤4,否则进行调整:\\
若 ,则找到信噪比余量最小的子载波组,假设其序号为,调整,继续步骤3直到;\\
如果,可跳到步骤4,否则进行调整:\\
若 ,则找到信噪比余量最大的子载波组,假设其序号为,调整,继续步骤3直到。\\
\item{\bf{步骤4:}}最后根据各个子载波上分配的比特数,按下式计算各子载波上应该分配的功率:
其中即为所求的第k个子载波上的功率,是表示步骤3中得到的各个子载波上分配比特数,是Fischer算法中的等效噪声方差,等于实际噪声方差除以信道增益,也就是本算法步骤1中信噪比SNR的倒数。
\end{description}

SBLA算法在初始比特分配和后面的比特调整中都是以子载波组为最小单位的,故在比特分配过程中其算法度是Fischer算法的1/M,M为每组中子载波数,并且在分配过程中不要进行对数运算,而其性能只是略低于Fischer等算法,其仿真结果将在后面给出。
\subsection{减少反馈信息的改进SBLA算法}
SBLA通过每组的方式进行比特分配,降低了这个过程的计算复杂度,并且减少自适应信令信息,除此之外,我们也可以针对可以光信道的特性进一步优化功率分配过程。在前面的SBLA算法中得到了比特分配之后,还是按照类似于Fischer算法对每个子载波进行功率分配,并且接收端要反馈所有子载波的功率值到发射端(假设自适应算法在接收端执行)。但是对于可见光低通信道信道的特点,在每个子载波组中,正常情况下功率分配总是随着频率增加而增加的,因为要想获得低误比特率,要尽量使得每组中各个子载波上的误比特率相等。所以基于这一特点,可以只根据式计算反馈每组子载波中第一个和最后一个子载波的功率,而其他子载波的功率可以使用线性插值得到:
其中表示一个子载波组中第i个子载波的功率,是反馈回来的每组中第一个和最后一个子载波功率。根据上式得到各个子载波功率之后再乘以一个系数使得总功率满足发射功率限制即可,我们称这种改进于SLAB的比特功率分配算法为Improved-SLAB算法。

\subsection{仿真结果分析}
图展示了SBLA、Improved-SBLA算法的BER性能。仿真中使用的信道是中的LOS信道,设置子载波总数为128,IFFT点数为256(还有128各共轭对此子载波),并且将0~3及124~127号共8个子载波人为设置为虚拟子载波,不传输数据,所以可用子载波为120个。在SBLA、ImprovedSBLA算法中将这120个子载波分为8组,每组含有15个子载波,目标速率为720 bit/OFDM signal,总功率设为128(与子载波总数相等)。从图可以看到SBLA、Improved-SBLA算法的BER性能相近,并且略劣于Fischer算法,但是较固定比特功率调制和人为挑选的而言,BER还是有较大的提高,这说明了Improved-SBLA算法对SBLA算法的改进是合理的,同时也充分体现了自适应相对于固定调制的优势。


图和图给出了设置信噪比为25 dB时在LOS信道下SBLA、Improved-SBLA算法比特和功率在各个子载波上的分配,并且也Fischer算法的结果进行了比较,可以发现SBLA、Improved-SBLA算法因为人为规定了分组,SBLA、Improved-SBLA算法的调制阶数变化只能发生在特定的子载波处,在比特分配上也Fischer算法存在较大的区别,但是还是有一些共性,即低频子载波分配了高阶调制而低频子载波分配低阶调制。从图中可以看到SBLA算法在每个子载波组上的是单调递增的,而Improved-SBLA则是利用了这一特性,使用了线性插值来得到功率分配,在图中也可以看出其功率分配在每个子载波组上是呈线性的,并且与SBLA功率分配的结果很接近,这也充分说明了Improved-SBLA算法的合理性。

\section{本章小结}
本章主要介绍了OFDM系统的比特和功率分配算法,研究其在可见光通信中的应用,选出了适合可见光通信的SBLA分配算法,并且在此算法的基础上进行了改进。首先阐述了自适应传输的理论基础—香农信息论和注水定理;然后说明了自适应传输的三种优化准则,即固定目标误比特率和发射功率的最大速率准则(RA)、固定目标误比特率和速率的最小发射功率准则(MA)及固定发射功率和速率的最小误比特率准则(BA),在此基础上介绍了OFDM自适应传输领域三个最经典的算法,分别是在RA和MA准则下最优的Hughes-Hartogs算法、BA准则下Chow算法和Fischer算法,详细说明了这些算法的推导和实现步骤,并且通过仿真比较了它们的性能差异,发现在可见光通信信道下它们在BA准则下BER性能相差不大;最后分析了适合子载波SNR相关性较大的SBLA算法,因为可见光信道本身就是低通的,天然合适SBLA算法的应用,并且进一步利用可见光通信信道特征,提出了适应线性插值来进行功率分配的Improved-SBLA算法,通过仿真发现改进的算法在减少了反馈量就运算复杂度的基础上,BER性能与SBLA相当,说明改进算法是合理可行的。
