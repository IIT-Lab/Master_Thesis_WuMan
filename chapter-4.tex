 % !Mode:: "TeX:UTF-8"
\chapter{可见光多波段OFDM系统速率自适应技术研究}
\section{引言}
自适应传输技术在上世纪60年代就已经被提出,其基本思想就是根据实时的信道质量决定调制参数,目标是优化通信质量,但是因为其计算复杂度很大,实现困难而没有引起研究人员足够的重视\cite{徐凌峰2007ofdm},直到上世纪80年代末,人们对高速可靠的通信系统的需求越来越强烈,同时由于数字集成电路的快速发展,其计算能力已能够支撑复杂的算法,所以自适应传输计算重新进入研究人员的视野,并且成功用于DSL、WCDMA等通信系统。前面我们已经介绍过OFDM系统及其信道估计的方法,已经了解OFDM技术是把实际通信信道划分成若干个子信道,每个子信道可以认为是独立传输的,如果所有的子载波上都使用同样的调制方式,那么整个系统的误比特率性能就由那些处于深衰落处的子载波决定,如在前两章已经介绍室内可见光信道就是低通的,则此时高频的子载波决定了整个系统的性能,这样的方法显然是不合理的,所以在第三章的仿真中就使用了表\ref{tab:modOrder}所示的调制策略,但是之前得到这样的策略是由主观判断得到的,而没有充足的理论依据。本章将详细介绍OFDM自适应技术及其在可见光通信中的应用,首先将从信息论的角度探讨自适应传输的原理,然后将介绍现有的几种经典的自适应算法并且仿真分析它们的性能,最后将介绍一种可见光自适应传输方案。
\section{自适应传输基本原理}
\section{OFDM系统自适应算法研究}
\section{可见光通信中的自适应方案研究}
\section{本章小结}