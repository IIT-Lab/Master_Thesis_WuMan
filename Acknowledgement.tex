% !Mode:: "TeX:UTF-8"
% 致谢

\begin{Acknowledgement}
%\fontsize{12pt}{14pt}\selectfont
时光荏苒,在硕士毕业论文即将完成之际,也意味着紧张、愉快且收获颇丰的研究生生涯就要结束,将要走向人生的下一段旅程,在此,请允许笔者对
曾经指导,教育和帮助过我的老师和同学们略表感情之情。

首先要衷心感谢我的指导老师赵春明教授,在三年的研究生学习中,赵老师的悉心指导让我受益终生。研一时,虽然已上课为主,但是赵老师还是每周
给我们开例会,让我们提前接触科研,为了我之后的发展指明方向;在研究生的第二年,赵老师又亲力亲为指导我们完成科研项目,他丰富的工程经验
帮助我们解决了很多难题,少走了许多弯路;到了研三找工作时,赵老师也为我们的职业发展提供了很多高价值的建议。如果没有赵老师一路来的引导
教育,整个研究生生涯将失色不少。

我也要特别感谢许威老师,许老师负责我研一时的具体指导工作,对我的科研工作进行了启蒙教育。正是许老师严格、细心而又不设限的指导方式让我
很快地进入了科研的节奏,并在研一时就取得了一点点成绩。同时许老师在科研学习中给我充分的自由,当我提出想做些工程方面的工作,许老师也坚
定不移地支持我的想法,让我有机会得到了工程方面的经验。许老师认真、热情、勤奋,与同学们亦师亦友,从他身上我学到了许多。

我还要感谢张华老师,姜明老师,黄鹤老师,梁霄老师在科研生活中给予我的指导和帮助。张华老师除了对我的科研学习上进行指导外,还经常跟我讨
论行业、产业发展的趋势,为我的职业选择提供了宝贵的意见;姜明老师负责实验室很多具体的事物,姜老师的辛勤付出,为了我们营造了一个舒心完
善的科研环境;黄鹤老师精通硬件逻辑编程,教授了我很多工程实现经验;梁霄老师擅长硬件设计,也曾给我很多指导和帮助。课题组的老师们都非常
认真负责,而又平易近人,是他们给了我一个完美的研究生学习经历。

特别感谢实验室的朱清豪,陶于阳,张俊,张立碧,吴宪,刘晶,王佳,邱朗,梁凌轩,官伟,凌欣彤同学,他们让整个实验室在学习过程中充满欢乐,
从他们每个人身上我都学习到了很多。和他们在一起学习和生活的日子,是我研究生生涯中最美好的回忆。感谢实验室成员在项目开发中的互助合作,
正是有大家的帮助才使得本论文得以顺利进行。

另外,我还要特别感谢我的父母和我的哥哥,他们是我最坚强的后盾和最温暖的港湾,我的每一点的进步和成绩都蕴含着他们的心血,也都属于他们。

最后,我还要感谢我的女朋友黄岚,是她伴我成长,给予我学习和生活中的诸多帮助,她也是我不断奋斗的动力源泉。
\begin{flushright}
吴满\\
2016年1月18日\\
\end{flushright}

\end{Acknowledgement}
