% !Mode:: "TeX:UTF-8"

% 中文摘要
\begin{abstract}{可见光通信\quad{}组网\quad{}最优化\quad{}灯组调度\quad{}增量式调度}

近年来,无线通信的发展受到了诸多的限制,如频谱资源的匮缺,射频对人体的辐射危害等。
而随着发光二极管(LED)技术的不断发展和在现代家庭中的不断应用,与之对应的使用LED灯进行数据传输的室内可见光通信技术也受到了极大的关注,并被进行了大量的深入研究。
可见光通信技术具有频谱资源丰富,绿色安全,能量转化率高等诸多优点,对于实现室内高速绿色通信具有重要的研究价值。本文主要着眼于室内可见光通信系统中的组网技术,解决典型室内灯组组网下多用户的高效通信问题。

首先,本文介绍了可见光通信及其组网技术的研究背景和研究进展,以及可见光通信系统中的基础理论。自从提出可见光通信的概念,可见光通信的研究就在全球各地广泛开展,但是目前研究较多的仍然是可见光通信的物理层技术领域,
以获得更高的信息传输速率作为目标。本文则着重介绍了可见光通信在组网技术方面的一些理论研究成果,并讨论了可见光通信系统中的链路方式,信道建模等内容。此外,本文还阐述了可见光通信在组网方面可能会遇到的一些技术问题。

其次,本文对可见光通信系统中LED灯组的最优化布局问题进行了研究。
对于该问题,本文并不是单纯地考虑覆盖的均匀度,而是以光照强度和光接收功率的均匀度作为一个目标,以接收平面的平均光接收功率值作为另一个目标,将其他条件作为约束,建立起一个多目标的最优化模型,
并通过将该多目标最优化模型转变为单目标最优化模型进行了求解。仿真结果表明,本模型可以通过合理地配置模型权重系数,以满足灯组布局的不同需求,并给出了对应需求下的光照强度分布和光接收功率分布图。

接着,本文研究了分布式灯组架构下的灯组调度问题。分布式灯组指每个灯组相当于一根天线,呈分布式分散在房间中。所有灯组都连接在一个调度器上,受该调度器统一控制。
在该架构下,本文提出了一种基于用户运动方向的灯组协同调度算法。该算法在灯组和用户进行数据通信时,测量用户的上行功率,利用一段时间内接收到的功率信息,获得用户相对于各灯组的运动方向。利用该方向信息,
将灯组分为用户前进方向灯组集和用户远离方向灯组集,在对用户前进方向灯组集进行强度补偿后,再进行灯组的调度,从而保证用户在室内移动时的可靠通信。
仿真结果表明,该灯组调度方法在数据重传方面的表现显著优于不使用灯组协作的调度算法和简单的灯组协同调度算法,可以满足系统对可靠通信的要求。

最后,本文还研究了独立式灯组架构下的灯组调度问题。独立式灯组则意味着每个灯组都具有控制功能,其天线在同一位置,灯组可以独立控制发送数据。
同样在该架构下,本文提出了一种增量式的灯组调度算法。该增量式的灯组调度算法可以分为两个部分,即长周期的全局调度和短周期的局部调度。
其中全局调度是针对所有的用户,以整体的系统容量为目标,将所有的用户划分在多个时隙中,确保每个时隙中的用户之间没有干扰。
而低复杂度的局部调度则是跟踪移动用户的运动,并调整之前产生的灯组调度结果以适应用户的移动。仿真结果表明,该增量式灯组调度算法对于多用户移动的场景,
可以在获得高系统容量的同时,显著降低了系统的调度复杂度。

\end{abstract} 