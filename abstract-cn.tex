% !Mode:: "TeX:UTF-8"

% 中文摘要
\begin{abstract}{可见光通信\quad{}OFDM\quad{}信道估计\quad{} 自适应传输\quad{} FPGA}
随着移动互联网的发展,其提供的各种方便快捷的功能已经深刻的改变了人们的生活方式。同时物联网、智能家居等相关产业也正处于快速发展阶段,如何为多种多样的智能设备提供安全可靠而又绿色经济的无线接入方式显得格外重要,虽然目前使用的蜂窝网及WIFI接入能解决大部分的接入问题,但是由于频谱资源的限制及射频辐射可能危害健康迫使人们寻找新的接入方式。同时发光二极管(Light Emitting Diode, LED) 作为新一代绿色高效的光源在照明市场上高歌猛进,预计到2021 年,其市场占有率将超过52\%。以LED为基础的室内可见光通信(Visible Light Communication, VLC)迎来了前所未有的发展机遇。本文对室内可见光通信领域的基于OFDM调制的自适应传输技术进行了研究,包括信道估计、自适应算法等,针对可见光信道特征,提出了性能不受影响下降低反馈量的简化比特功率分配算法。

首先,本文介绍了可见光通信的研究背景,包括其应用场景及与传统通信方式相比的优势,同时也对可见光通信在国内外的发展历程进行了概述。

第二章首先可见光通信的基本原理,包括系统模型、信道特征及光电元器件等,本文联系实际情况将可见光通信信道建模为线性信道,然后介绍了电光转换器件LED,详细说明了荧光激发型LED和多色混合型LED的工作原理和特性,还介绍PIN 和APD两类光电转换器件及接收端使用的滤光片。理解这些光电元器件的原理与作用有益于理解整个可见光通信系统。由于OFDM调制技术非常适应光低通信道,在可见光通信中也得到了广泛的使用,所以本章也介绍了光OFDM技术,主要说明了DCO-OFDM和ACO-OFDM的工作原理及区别,在此基础上提出了一种改善ACO-OFDM 系统PAPR 性能的RoC-ACO-OFDM方案,并且在理论和数值仿真的角度论证了该方案的有效性。

第三章主要分析了OFDM信道估计方法的基本原理及其在可见光通信中的应用,首先探讨了OFDM信道估计的常用方法,重点放在基于导频的方法中,系统研究了基于最小二乘法的LS信号估计方法、基于最小均方误差的MMSE方法及其基于MMSE两个改进方法LMMSE和SVD分解方法;然后结合可见光通信系统设计的实例,使用ZC序列作为导频,通过仿真的方法比较了上述方法在可见光信道下的性能,发现虽然LS 信道估计方法在本系统工作点(SNR=25 dB)附近性能稍逊于LMMSE及SVD算法,但是其实现要简便得多,所以在本课题硬件设计中将采用LS进行信道估计;最后讨论了OFDM系统中信噪比的估计方法,分析了基于导频的估计方法和EVM方法,得出了EVM方法虽然需要额外的开销,但是其估计值更加吻合实际系统,推荐在可见光自适应传输系统中使用该方法。

第四章主要介绍了OFDM系统的比特和功率分配算法,研究其在可见光通信中的应用,选出了适合可见光通信的SBLA分配算法,并且在此算法的基础上进行了改进。首先阐述了自适应传输的理论基础—香农信息论和注水定理;然后说明了自适应传输的三种优化准则,即固定目标误比特率和发射功率的最大速率准则(RA)、固定目标误比特率和速率的最小发射功率准则(MA)及固定发射功率和速率的最小误比特率准则(BA),再次基础上介绍了OFDM自适应传输领域三个最经典的算法,分别是在RA和MA准则下最优的Hughes-Hartogs算法、BA准则下Chow算法和Fischer算法,详细说明了这些算法的推导和实现步骤,并且通过仿真比较了它们的性能差异,发现在可见光通信信道下它们在BA准则下BER性能相差不大;最后分析了适合子载波SNR相关性较大的SBLA算法,因此可见光信道本身就是低通的,天然合适SBLA算法的应用,并且进一步利用可见光通信信道特征,提出了适应线性插值来进行功率分配的Improved-SBLA算法,通过仿真发现改进的算法在减少了反馈量就运算复杂度的基础上,BER性能与SBLA相当,说明改进算法是合理可行的。

第五章在前面几章讲解可见光通信原理及其自适应传输理论的基础上,介绍了本课题对应的硬件平台。首先对整个硬件平台进行了概述,包括各种器件的参数及其选择依据,并且简述了发射端和接收端的基带处理处理过程;然后重点描述了基带处理中与自适应技术密切相关的调制器、自适应参数生成、软解调三个模块;最后展示了整个硬件平台实物图。

第六章对全文进行简要的总结并分析该领域之后的研究方向。

\end{abstract}
