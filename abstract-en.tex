% !Mode:: "TeX:UTF-8"

% English Abstract
\begin{englishabstract}{Visible Light Communication\quad{}OFDM\quad{}Channel Estimation\quad{}Adaptive Transmission\quad{}FPGA}

With the development of mobile Internet, which provides a variety of convenient features and has profoundly changed people's life. Meanwhile, Internet of things, smart home and other related industries are also in rapid development stage, how to provide reliable and secure wireless access to these intelligent devices in a green economy way is particularly important, although the use of cellular and WIFI can solve most of the access problem, but due to spectrum limitation and potential health problem caused by radiation forcing people to look for new technology. At the same time as a new generation of green efficient light source, LED is very popular in the lighting market,  its market share is expected be more than 52 \% in 2021. LED-based indoor visible light communication meets an unprecedented development opportunity. In this paper, adaptive transmission in the field of VLC based on OFDM modulation techniques are studied, including channel estimation, adaptive algorithms and so on.

At first, This paper introduces the research background and summarizes the development history of visible light communication. Then briefly introduces the basic principles of visible light communication, including system model, the channel characteristics, electro-optical and optical-electro conversion device, etc. According to the actual situation, visible communication channel is modeled as a linear channel in this paper. The optical OFDM is also studied, including the DC-biased optical OFDM(DCO-OFDM) and Asymmetrically clipped optical OFDM(ACO-OFDM). Then, a peak to average power ratio(PAPR) reduction technology for ACO-OFDM called RoC-ACO-OFDM is proposed. 

Subsequently, the common OFDM channel estimation methods are discussed, including LS, MMSE, LMMSE and SVD. Then, we compares the performance of these estimation algorithms using ZC sequence as pilot,  the same as in our demo system. The result shows that the performance of LS method is slightly inferior to LMMSE and SVD algorithm, but its implementation is much more simple. The SNR estimation methods are discussed at last. Pilot-based and EVM methods are analyzed. we find that EVM is much more accurate and suggest this approach in our real system design.

Then, this thesis discusses the bit and power allocation algorithm for OFDM systems. The Shannon information theory and water filling algorithm are given at first as fundamental knowledge of adaptive transmission. Then introduces three most classic allocation algorithms, they are optimal Hughes- Hartogs algorithm, Chow algorithm and Fischer algorithm. The performance of these algorithms is examined through simulation under wireless optical channel. The result reveals that they have a similar performance. At last, simple block loading algorithm(SBLA) is analysed , which is very appropriate for wireless optical channel. A improved algorithm based on SBLA is proposed aiming to reduce feedback information from receiver to transmitter.

Finally, based on the visible light communication principle and adaptive transmission theory explained in the previous chapters, we introduces our  hardware demo platform. First, an overview is given to the entire hardware platform, including the parameters of a variety of devices and the reason why choose them. Then, the baseband processing at both transmitter and receiver end is described. Subsequently, the modules which are related to adaptive transmission is analyzed in detail, including modulation module at transmitter end and demodulation module and adaptive parameter generator module at receiver end. A presentation of our demo system is given out at last.

\end{englishabstract}
