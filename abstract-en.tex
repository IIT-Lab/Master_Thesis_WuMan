% !Mode:: "TeX:UTF-8"

% English Abstract
\begin{englishabstract}{Visible Light Communications\quad{}Networking Technology\quad{}Optimization\quad{}LED Scheduling\quad{}Incremental Scheduling}

In recent years, the development of wireless communication has encountered many restrictions,
such as the scarcity of spectrum resources, the radio frequency (RF) radiation hazards to human body.
With the continuous development of light emitting diode (LED) technology and the continuous application of LED in the modern family,
the corresponding indoor visible light communication (VLC) technology, which uses the LED lights for data transmission,
has attracted a great attention and was carried out in-depth research.
For VLC has great advantages in rich spectrum resources, green and safe use, high energy conversion rate and other aspects,
it has important research value for achieving green high-speed indoor communication.
This paper mainly focuses on networking technology of indoor VLC systems to efficiently solve the multi-user communication problem in typical indoor networking scenarios.

Firstly, the paper introduces the research background and research progress of VLC and networking technology of VLC.
Since the concept of VLC was proposed, studies of VLC have been widely carried out in the world, but most researches focus on physical layer of VLC to achieve a higher rate of data transmission.
However, the paper focus on the theoretical research in networking technology of VLC systems. In addition, the paper also introduces the related basis theories of VLC,
 such as optical communication link mode, channel analysis, and the paper describes some technical issues in networking terms we may encounter.

Secondly, the layout optimization problem of LED lights of VLC systems has been studied.
For this problem, the paper is not simply considering the uniformity of coverage,
but establishes a multi-objective optimization model with light intensity and light uniformity of the received power as a target, with the average power value of the receiving plane as another target,
with the other aspects as constraint conditions. And the multi-objective optimization model can be transformed into a single-objective optimization model to solve.
Simulation results show that the model can be configured through the weighting coefficients to adjust the LED layout for the different needs and achieve a reasonable optimization layout results.

Thirdly, the paper studied scheduling problem of VLC systems with distributed LED lights.
Distributed LED lights mean that all the lights are connected to a scheduler, and controlled by the scheduler.
In the framework, we propose a user-direction based collaborative LED scheduling scheme.
As the scheme shows, user's uplink power can be measured by the LED arrays through data communication of user and LED arrays, thus the user's direction information can be calculated by a period time of power measurement.
Considering the direction informations above, LED arrays can be scheduled to fit the user movements, thus to ensure the reliable communication when user moves in the indoor environment.
Simulation results show that the proposed scheme has better performance than the scheduling scheme without using collaborative LED lights and simple cooperative LED lights scheduling algorithm in data retransmission aspects.
The proposed scheme can also meet the system requirements for reliable communications.

Finally, the paper also studied scheduling problem of VLC system with independent LED lights.
Independent LED lights mean that lights can independently send data to user without affecting other LED arrays.
We propose an incremental scheduling scheme in this framework. The incremental scheduling scheme can be divided into two phases, namely the long-term global scheduling and short-term local scheduling.
The long-term global scheduling, which is scheduled for all users, is used to eliminate inter-user interference and maximize system capacity,
while short-term local scheduling, which is scheduled only for moving users by low complexity scheme, can eliminate the inter-user interference caused by user movements and reduce the computation.
The simulation results show that the incremental scheduling scheme can work efficiently in the multi-user motion scenarios, and can achieve near-optimal performance, but with low complexity.

\end{englishabstract}
